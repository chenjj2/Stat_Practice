\documentclass[a4paper,10pt]{article}

\usepackage[english]{babel}
\usepackage[T1]{fontenc}
\usepackage[ansinew]{inputenc}

\usepackage{lmodern}	% font definition
\usepackage{amsmath}	% math fonts
\usepackage{amsthm}
\usepackage{amsfonts}

\usepackage{tikz}
\usetikzlibrary{shapes,arrows}
\usetikzlibrary{calc}

%%%<
\usepackage{verbatim}
\usepackage[active,tightpage]{preview}
\PreviewEnvironment{tikzpicture}
\setlength\PreviewBorder{5pt}%
%%%>

%\usetikzlibrary{decorations.pathmorphing} % noisy shapes
\usetikzlibrary{fit}					% fitting shapes to coordinates
\usetikzlibrary{backgrounds}	% drawing the background after the foreground

\begin{document}

\begin{figure}[htbp]
\centering
% define hyper shape
\tikzstyle{hyper}=[draw, ellipse,
                                    minimum height=1.2cm,
                                    minimum width = 1.8cm,
                                    fill=yellow!100]
% define local shape
\tikzstyle{local}=[draw, ellipse,
				minimum height=1.2cm,
				minimum width = 1.8cm]

% define data shape
\tikzstyle{data}=[draw, ellipse,
                                    minimum height=1.2cm,
                                    minimum width = 1.8cm,
                                    fill=gray!50]

% define data shape
\tikzstyle{blank}=[draw, ellipse,
                                    minimum height=1.2cm,
                                    minimum width = 1.8cm,
                                    draw = gray!0,
                                    fill=gray!0]


% define background rectangle
\tikzstyle{background}=[rectangle,
					draw = gray!100,
					inner sep=0.4cm]

% define special arrow
\tikzset{
  -|-/.style={
    to path={
      (\tikztostart) -| ($(\tikztostart)!#1!(\tikztotarget)$) |- (\tikztotarget)
      \tikztonodes
    }
  },
  -|-/.default=0.3,
  |-|/.style={
    to path={
      (\tikztostart) |- ($(\tikztostart)!#1!(\tikztotarget)$) -| (\tikztotarget)
      \tikztonodes
    }
  },
  |-|/.default=0.3,
}

% draw
\begin{tikzpicture}[node distance = 0.8cm, >=latex,text height=1.5ex,text depth=0.25ex]

  % The various elements are conveniently placed using a matrix:
  \matrix[row sep=1.5cm,column sep=0.5cm] {
    % First line
        \node (alpha) [local] {$\alpha$}; &
        \node (offset)   [hyper] {$C$};     &
        \node (slope) [hyper] {$S^{(1-4)}$}; &
        \node (scatter) [hyper] {$\sigma_R^{(1-4)}$}; &
        \node (transition) [hyper] {$T^{(1-3)}$};
	\\
    % Second line          
        \node (merr) [data] {$\Delta M^{(i)}_{ob}$}; &
        \node (mt)   [local] {$M_t^{(i)}$};     &
        \node (rt) [local] {$R_t^{(i)}$}; &
        \node (rerr) [data] {$\Delta R^{(i)}_{ob}$}; &
        \node (2ndright) [blank] { };
        \\
    % Third line
        &
        \node (mob) [data] {$M_{ob}^{(i)}$}; &
        \node (rob)   [data] {$R_{ob}^{(i)}$};  &
        \node (bottomright) [blank]{$N$};
        \\
    };
    
    % The diagram elements are now connected through arrows:
    \path[->]
        (alpha) edge (mt)
        
        (mt) edge (rt)
        (slope) edge (rt)
        
        (merr) edge (mob)
        (mt) edge (mob)
        (rerr) edge (rob)
        (rt) edge (rob)        	              
        ;
    
    \draw[->] (offset.south) to[|-|] (rt.north);
    \draw[->] (scatter.south) to[|-|] (rt.north);
    \draw[->] (transition.south) to[|-|] (rt.north);
 
 % rectangle
    \begin{pgfonlayer}{background}      
        \node [background,
                    fit=(merr) (bottomright),
                    ] {};
    \end{pgfonlayer}
    
    
\end{tikzpicture}

\end{figure}



\end{document}